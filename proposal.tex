% !TEX program = xelatex
\documentclass[a4paper]{article}

% --- 中文設定,使用 xeCJK 套件 ---
\usepackage{xeCJK}
\setCJKmainfont{PingFang TC} % 設定中文字型,蘋方是macOS內建的好選擇

% --- 基本文件資訊 ---
\title{114-1 資料庫管理\ 期末專案}
\author{資管三~羅立宸~黃元翔~陳以倫}
\date{\today} % \today 會自動抓今天的日期

\begin{document}

\maketitle % 產生標題

\section{系統分析} % 這是第一節的標題
學期初買的教科書超貴,學期末卻堆在角落長灰塵?想買學長姐的二手書,卻不知道去哪裡找?如果你有這些困擾,趕緊上「BookSwap」尋找你需要的二手好物!

「BookSwap」是一個提供給某大學學生刊登及尋找二手教科書與物品的平台 ,主要目的是幫助該校學生解決二手物品(特別是教科書)資訊不流通、交易媒合困難的窘境 。平台上的「刊登」是指一次性的物品出售貼文,每篇貼文都是獨立的,有自己的刊登編號 。交易完成或下架後,該刊登即失效。使用者可以主動舉報可疑貼文或留言,建立社群自我管理的基礎。每次違規都會被記錄,達一定次數(例如 5 次)後,系統會自動停權或封鎖帳號,防止惡意刊登或詐騙帳號重複出現。

根據不同的功能及掌控權限,「BookSwap」系統的用戶可以分為兩種身分,分別是 User 及 Admin 。
User(一般使用者):可依照自身需求,選擇「刊登」想賣的物品,或是「瀏覽/搜尋」平台上由其他人刊登的物品 。若想刊登物品,可透過介面輸入物品的標題、描述、價格、分類,並上傳照片 。如果物品是教科書,還可以關聯到特定「課程」 。如果使用者想購買物品,則可瀏覽平台上現有的刊登,並選擇感興趣的物品進行聯繫。
Admin(管理者):則是「BookSwap」系統的業務經營者,主要負責管理「課程」及「物品分類」的資訊 ,並且可查詢所有使用者的刊登紀錄,收到User舉報後會審核、決定是否移除或警告不適當的刊登內容 。
\subsection{系統功能}
\subsubsection{關於刊登的相關設定}
系統會提供「物品分類」讓使用者選擇,例如:教科書、3C 產品、生活用品等。如果使用者選擇的分類是「教科書」,系統會建議使用者從一個課程列表中選擇想關聯的「課程」 ,例如「資料庫管理」,以利他人搜尋。使用者在刊登時,可以上傳多張物品照片(例如,至多 5 張)。每則刊登都會有「狀態」,例如:刊登中、預訂中、已售出 。

\subsubsection{給 User 的功能}
在本系統中,User 可以執行以下功能:
\begin{itemize}
    \item 新增刊登:使用者能透過設定物品標題、描述、價格、分類、上傳照片等相關資訊來刊登一項物品 。如果物品是教科書,還可以選擇想關聯的課程 。一旦刊登,系統便會給定一個屬於該刊登的編號 。
    \item 新增留言:使用者能透過在刊登的底下留言,例如"想要"、"有興趣"來表達自己的意願,也能使刊登者知道自己刊登的物品是否有人有意願。
    \item 收藏物品:使用者若看到感興趣的刊登,可將其「加入收藏」,作為一筆新的收藏資料新增至資料庫 。
    \item 管理刊登:使用者如果不想繼續販售,可刪除(或下架)自己刊登的物品 。
    \item 查詢使用者曾經刊登過的物品:使用者可以查詢自身曾建立過的刊登和刊登的所有留言,包括「刊登中」與「已售出」的 。
    \item 查詢使用者收藏的物品:使用者可以查詢自身收藏過的刊登 。
    \item 查詢目前平台上的刊登:使用者可依分類、課程、關鍵字或價格,查詢尚未售出且刊登中的物品 。
    \item 舉報不當刊登:若使用者發現疑似詐騙、違禁品或不當內容的貼文,可透過「舉報」功能提交檢舉。
\end{itemize}

\subsubsection{給 Admin 的功能}
在本系統中,Admin 可以執行以下功能:
\begin{itemize}
    \item 管理課程:業務經營者可對課程資訊進行增刪改查的操作 ,以確保課程列表是最新狀態。
    \item 管理分類:業務經營者可對物品分類進行增刪改查的操作 。
    \item 查詢使用者資訊:業務經營者可查詢所有使用者的活動紀錄,包括該使用者曾經刊登過哪些物品 。
    \item 查詢刊登資訊:業務經營者可查詢所有刊登的詳細資訊,並移除違規(如詐騙或違禁品)的刊登 。
    \item 審核舉報與處理違規:業務經營者會審核舉報內容,若經查屬實,將移除該刊登並記錄違規行為。當使用者違規達一定次數後,系統會自動封鎖該使用者帳號,以維護平台秩序與安全。
\end{itemize}

\section{系統設計}
\subsection{ER Diagram}
\subsection{Relational Database Schema Diagram}
\subsection{Data Dictionary}
\subsection{正規化分析}


\end{document}
